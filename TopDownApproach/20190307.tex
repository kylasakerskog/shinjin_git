\documentclass[9pt,a4j,twocolumn]{jsarticle}
\usepackage[dvipdfmx]{graphicx}
\usepackage{ascmac}
\usepackage{txfonts}
\usepackage{float}
\usepackage{cite}
\usepackage{listings}

\setlength{\topmargin}{-3cm}
\setlength{\textheight}{29.5cm}
\setlength{\textwidth}{20.0cm}
\setlength{\oddsidemargin}{-0.9cm}
\setlength{\evensidemargin}{-0.9cm}
\setlength{\baselineskip}{0.5cm}
\renewcommand{\baselinestretch}{0.865}

\title{
    Computer Networking\\
    A Top-Down Approach pp.623-656
}
\author{
    %トップダウン・アプローチの作者
    James F. Kurose, Keith W. Roth
    \\\\
    情報工学科3年 寺岡研究室\\
    61619027 安森 涼 
}
\date{2019年 3月 7日}
\begin{document}
\maketitle

\section{即時性のある対話型アプリケーションプロトコル}
VoIPやビデオ会議を即時性のある対話型アプリケーションは現在,頻繁に利用される.IETFやITUがこれらのアプリケーションに対する標準を形作っている.もし適切な標準化がされていれば個々の会社がそれぞれ相互運用できる新しい製品を開発することができる.この章ではRTPとSIPの標準について述べる.
\subsection{RTP}
前回,VoIPアプリケーションの送信側でトランスポート層通過前に音声チャンクにヘッダフィールドを加えると述べた.ヘッダフィールドにはシーケンス番号とタイムスタンプが含まれ,音声動画ファイルとともに標準化する.RTPはRFC3550で定義されているものが標準である.RTPはPCM,ACC,MP3形式の音声とMPEG,H.263形式の動画の送信に使われる.このように幅広い実装に役立っている.この節ではRTPの紹介をおこなう.

RTPはUDPの上部で動く,送信側はメディアチャンクをRTPパケットにカプセル化し,UDPセグメントにカプセル化しIPに渡し,受信側がRTPパケットをUDPセグメントから取り出し,メディアチャンクをRTPパケットから取り出す.そして,チャンクをメディアプレイヤに送信し,復号化とレンダリングをおこなう.例として音声をRTPで送信することを考える.音声は64kbpsのPCMで符号化されている.チャンクは20msec(160kbyte)ごとに送られる.送信側はそれぞれのチャンクにRTPヘッダを加える.RTPヘッダには音声符号化の形式,シーケンス番号とタイムスタンプを含み,基本的には12byteで構成されている.チャンクとRTPヘッダを合わせたものをRTPパケットと呼ぶ.これを上記のようにカプセル化して送信し,受信側で取り出していることで送受信をおこなっている.このRTPヘッダにより,他のネットワーク接続されたマルチメディアアプリケーションにと相互運用できる.

RTPは時間通りに送信を行うことおよび,その他のQOSを保証していない.また,パケットの配達と,順序が異なる配達を防ぐことですら保証していない.事実,カプセル化はエンドシステムでのみおこなわれ,ルータはRTPを送信するIPデータグラムか否かを判別できない.RTPは個別のパケットを個々のパケットのストリームに割り当てる.例えば,2人がビデオ会議をする際に,音声と映像について,それぞれ両方向に2つの流れ,計4つのストリームを割り当てる.しかしながらMPEG1やMPEG2は音声と映像を一つのストリームに割り当てるので両方向計2つのストリームで大丈夫である.RTPではセッションと呼ばれる参加者のグループが規定されており,参加者ごとのセッションの識別には、ネットワークアドレス、データを送信するポートの組、データを受信するポートの組が使用される.参加者は複数のセッションに同時に参加することも可能である.

\subsubsection{RTPヘッダフィールド}
RTPヘッダフィールドにはペイロードタイプ,シーケンス番号,タイムスタンプ,SSRC識別子が含まれている.ペイロードタイプは7bitで.音声,映像符号化の形式が含まれている.これにより受信側に通知する.続いて,シーケンス番号フィールドは16bit長でRTPが送られるたびにインクリメントを行い,受信者にパケットロスを通知し,パケットを回復する.86-89のシーケンス番号を送った時に87,88のパケットがロスしたと受信者に通知されたら,復旧させるように試みます.
タイムスタンプフィールドは32bit長で,受信側がネットワークで生成されたパケットの乱れを除去する,また,同期playoutを受信側に提供している.
SSRCはは32bitでRTPストリームのソースを識別するのに利用される.RTPセッションは区別されたSSRCを持っている.SSRCは送信側のIPアドレスではなく,新しいストリームが作られるたびにランダムに番号を割り振る.2つのストリームに同じ番号が割り振られるのは稀だが,起きた時は新しい番号を振り直す.

\subsection{SIP}
SIPはオープンで軽いプロトコルで,IPネットワークをもとに,発信者と着信者の間の通信でコールを確立する機構である.発信者が着信者に発信開始および終了の合図を行う.また発信者に受信者の現在のIPアドレスを通知している.また,コール管理の機構を提供している,コール中に新しいメデイアストリームを加える,符号化を変更,コールに新しく参加者を加入させることとコール転送と,コール保留を行う.

\subsubsection{既知のIPアドレスとのコール確立}
SIPについて具体的に説明する.アリスがPCでボブのPCにコール設立をしたいとする.彼らのPCにはコールの送受信のSIPに基づいたソフトウェアが装備されているとする.



\end{document}